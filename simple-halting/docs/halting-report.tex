\documentclass[12pt]{article}

\usepackage{../../sbc/sbc-template}

\usepackage[english]{babel}
\usepackage[utf8]{inputenc}
\usepackage[T1]{fontenc}
\usepackage[colorlinks]{hyperref}
\usepackage{amssymb, amsmath, algorithm, algorithmic, color, listings}

\definecolor{codegreen}{rgb}{0, 0.6, 0}
\definecolor{codegray}{rgb}{0.5, 0.5, 0.5}
\definecolor{codepurple}{rgb}{0.58, 0, 0.82}

\lstset{
  basicstyle=\tiny\ttfamily,
  breaklines=true,
  commentstyle=\color{codegreen},
  extendedchars=true,
  keywordstyle=\color{blue},
  numbers=left,
  numberstyle=\tiny\color{codegray},
  showstringspaces=false,
  stringstyle=\color{codepurple}
}

\sloppy

\renewcommand{\algorithmiccomment}[1]{\hfill $\triangleright$ #1}
\renewcommand{\algorithmicrequire}{\textbf{Input:}}
\renewcommand{\algorithmicensure}{\textbf{Output:}}

\title{Solution to a Simplified Halting Problem\footnote{
    As submitted to the INE410113 class (Theory of Computation).
    \href{https://github.com/zambonin/ine410113}{Source code}.}}

\author{Douglas Martins\inst{1}, Emmanuel Podestá Jr.\inst{1}, Gustavo
    Zambonin\inst{1}}

\address{Departamento de Informática e Estatística, Universidade Federal de
    Santa Catarina \\
  88040-900, Florianópolis, Brazil % chktex-file 8
  \email{\{marcelino.douglas,emmanuel.podesta,gustavo.zambonin\}@posgrad.ufsc.br}
}

\begin{document}

\maketitle

\section{Introduction}\label{sec:intro}

The halting problem is well known in computability theory. In 1937, a
groundbreaking result related to this problem was proved by
Turing~\cite{Turing:article:1937:jan}. In modern terms, the proof states that
an algorithm, which use is to determine whether a Turing machine processing a
given program stops, cannot exist. More precisely, this problem can be defined
as a decision problem and determines if a program will continue running or not
based on a generic machine-program input pair. Formally, it can be defined as
follows. Consider a Turing machine, that accepts as input a machine $M$ and an
aptly encoded program $w$, \emph{i.e.} $Halt_{TM} = \{(M, w) \mid M
\text{ halts with } w\}$. Additionally, let $H$ be a machine that solves
$Halt_{TM}$ if, for any input $(M, w)$, $H$ either accepts or rejects.

We cannot build such a machine $H$, hence $Halt_{TM}$ is undecidable. We give a
short proof in the sequence. Suppose that machine $H$ exists and decides
$Halt_{TM}$, and let <$A$> be a codification of a Turing machine $A$. We
introduce a new machine $B$ that takes $A$ and runs $H$ with $(A,
\text{<}A\text{>})$, and halts if and only if $H$ rejects. Moreover, consider
that $B$ receives as input <$B$>, and will run $H$ with $(B,
\text{<}B\text{>})$. This scenario has two outcomes: (i) if $H$ accepts, then
$B$ halts on input <$B$>. However, by definition, $B$ will enter in loop if $H$
accepts. Hence, $B$ on input <$B$> does not halt; and (ii) if $H$ rejects, then
$B$ does not halt on input <$B$>. However, by definition, $B$ will terminate if
$H$ rejects. Hence, $B$ halts on input <$B$>. Therefore, the halting problem is
undecidable.

As an alternative, we can use reduction to prove that a problem is undecidable.
More precisely, a problem $A$ is reducible to $B$ if $B$ can be used to solve
$A$. Hence, if $A$ is undecidable, is enough to show that a solution in $B$ can
be used to decide $A$, unveiling a contradiction, and thus demonstrating that
$B$ is undecidable. Based on this property, if the halting problem was shown to
be decidable, several problems would also be decidable. For instance,
Goldbach's conjecture and calculation of the Kolmogorov complexity of a
program.

We can simplify the halting problem to transform it into decidable, by means of
adding restrictions to the possible machine-program pairs. In the following
sections, we will define and solve an instance of a simplified halting problem.
Moreover, details about the implementation and a proof of the problem
decidability will be unveiled.

\section{Simplified Halting Problem}\label{sec:halting}

It is known that there exists no algorithm which can determine if any program
halts or not, as seen above. However, there exist deciders that can know if
strings belong to a certain formal language. This is exactly the set of
recursive languages. For instance, every context-free language is
decidable~\cite[Theorem 4.9]{Sipser:book:2012}. Many other examples of
languages that always halt exist. Such is the case of what we call the
Simplified Halting Problem (SHP), described by
Demasi~\cite{Demasi:misc:2013:may} in an assembly-like format. We briefly cover
its details and prove that it is decidable after.

The main characteristics of SHP lie in their limited syntax and arithmetic
operations. Commands are divided into three groups: (i) usual conditional
branches that compare two integers and a copy operation; (ii) arithmetic
calculations modulo $1000$, with the additional restriction that dividing and
taking the modulo of zero is equal to zero itself; (iii) looping constructs in
the form of a \texttt{CALL} command, and program return \texttt{RET}, that
accept one argument. Further, there are ten registers that compose the
``memory'', where \texttt{R0} and \texttt{R9} receive input and place the
output from the third group of commands, respectively. Finally, programs have
an upper bound of a hundred lines.

Note that there is only one method of looping, that is, recursive. This is
exactly what the \texttt{CALL} command achieves. As such, to execute programs
of this language, it is imperative to maintain a recursion stack and ``program
counters''. Moreover, all ``memory'' is preserved between recursion calls,
except for the last register. \texttt{R1} to \texttt{R8} can be seen as
scratchpad registers. We present below an example of source code for SHP and
execute it step-by-step. Note that syntactic limitations such as delimitation
of conditional branches and uppercase enforcement is not discussed. The first
line informs that the program has $8$ lines and that its first input is $9$.

{\footnotesize
\begin{verbatim}
8 9
IFEQ R0,0
RET 1
ENDIF
MOV R1,R0
SUB R1,1
CALL R1
MUL R9,2
RET R9
\end{verbatim}
}

We will start by comparing equality of $\texttt{R0} = 9$ and $0$ and returning
$1$ if so. This is the stop condition of the recursive loop. Then, we copy
\texttt{R0} to \texttt{R1}, and decrement \texttt{R1}. We save the ``memory''
state and start a new execution of the program with $\texttt{R1} = 8$ as
\texttt{R0}. This is repeated until the stopping condition is reached, and
thus, the recursion stack has nine entries. When \texttt{RET 1} is executed,
$\texttt{R9} = 1$ and the ``program counter'' moves to the \texttt{MUL}
instruction, since it is directly after \texttt{CALL}. Then, \texttt{R9} is
doubled and returned, once again to the line below \texttt{CALL}, since it was
the origin of the recursion.

We can see that this is repeated as long as there are entries in the stack.
Since the stopping condition returns $1$ and that is doubled repeatedly, the
point of this program is to recursively calculate the $n$-th power of $2$,
where $n$ is the recursion depth and, indeed, the input of the program.

Now, we turn to the question on whether SHP is decidable. Evidently, the
language has a finite number of configurations, due to its limited arithmetic
operations and memory bank. This behaviour is independent of the instruction
that modified the program state, since a given configuration can be reached
from different lines. By this reasoning, it is enough to check that the number
of lines executed is greater than the number of possible configurations. This
is akin to say that the language is as powerful as a linear bounded automaton.

This strategy, albeit correct, will not yield the necessary performance on
simple programs such as \texttt{CALL R0; RET R0}. The state for this program
actually never changes, and still, to identify its non-halting behaviour, one
would have to go through all the possible program configurations. However, note
that if a configuration is reached more than once, this means exactly that an
infinite loop exists. That is, a state that leads to itself will never exit
this connection. The pigeonhole principle is the formalisation of this notion.

Identifying state repetitions is a simpler approach, since it needs not to look
at all the possible states for a program. Since it cannot have more than $100$
lines, the only instruction that can alter this behaviour semantically is
\texttt{CALL}, by means of its looping capabilities. Hence, infinite loops only
happen when two configurations are equal, exactly when those instructions are
executed. Thus, it is only necessary to check for situations of this case.

With this information in mind, we need only to define what is such a
configuration in the context of SHP\@. Recall that no additional information on
the depth level or line number is needed. Then, we define the ``memory state'',
that is, the sequence of registers \texttt{R0} to \texttt{R9}, as the
configuration. By means of checking if a ``memory state'' has already occurred
within former \texttt{CALL} instructions, one can identify if an infinite loop
will occur. Ergo, it is always possible to know whether a program will halt or
not. Thus, SHP is decidable.

\section{Implementation}\label{sec:imp}

The full source code for the SHP language parsing and execution unit can be
read in Appendix~\ref{app:impl}, stripped of comments for brevity. We shortly
discuss it in the sequence. Data structures were carefully chosen to prevent
data access overhead. As such, unordered maps and sets are heavily featured,
since their underlying STL implementation is a red-black tree. Additionally,
hashing functions were created, to enable insertion of custom structures into
these containers.

We parse the input with regular expression rules created from the description
of the problem~\cite{Demasi:misc:2013:may} and check for any syntax errors. We
must note that, while the program description says that arithmetic is limited
to modular arithmetic modulo $1000$, there exists a wrong input program that
features exactly this number. Hence, the program has a workaround to accept
this variable. Further, the input value limit of $100$ is also not respected
--- we believe it is a typo and should be $0 \leq n < 1000$. Helper functions
that translate values from the ``memory'' and parse each program line into a
valid ``instruction'' are given shortly below.

Lines 115--159 present the function that loops over an entire program and
evaluates its lines, following the control flow if needed. It takes into
account a ``program counter'' and a recursion stack, essential to simulate the
inner works of the toy language. If instructions are arithmetic or conditional,
a function map is invoked to decide what should be executed. Further, in the
case of arithmetic functions, recall that operations have to follow a set of
rules that, for instance, allow division by zero.

The most complex behaviour of the code, however, is met when parsing
instructions that enter and exit recursion states. An optimisation is added
here, that features memoisation of previously calculated recursion states. This
allows the code to pass the $\frac{1}{8}$ of a second time target, since there
are input programs that calculate large Fibonacci numbers, and it is known that
if intermediary results are not memoised, this calculation will take a very
long time. Evidently, the recursion stack is modified, the ``program counter''
is reset, and values are returned as usual.

Within this code, the strategy for preventing infinite loops is also
implemented. It is enough to compare if a recursion state has already occurred,
as discussed above. Note that the hashing function used in this comparison was
not carefully constructed as to prevent collisions, but it seems to distribute
all ``memory'' states distinctly. Finally, once the instruction is parsed, the
``program counter'' is incremented and the program may return if it finished
its execution.

All programs are executed and results are printed according to the problem
definition. After determining its correct parsing and execution of the toy
language, the code takes $30$ ms to calculate the entirety of the input.
Parsing instructions uses a considerable amount of time, but this cannot be
memoised, \emph{e.g.} \texttt{ADD R1,R2} may appear in two programs, but have
completely different meanings depending on the contents of the ``memory'' at
each ``program counter'' state.

Using simpler string structures, manual translation of strings to integers, and
substituting the ``memory'' from a map to an array lead the code to execute in
$4$ ms. Correctly emulating recursion loops, as well as these performance
improvements, were the main obstacles. Further optimisations can be achieved if
large parts of the code is rewritten, for instance, removing function maps and
regular expressions. Ergo, to prevent reduction of code readability and
maintenance, we leave it as is.

\bibliographystyle{../../sbc/sbc}
\bibliography{ref}

\appendix

\section{C++ implementation of SHP}\label{app:impl}
\lstinputlisting[language=C++]{../halting.cpp}

\end{document}
