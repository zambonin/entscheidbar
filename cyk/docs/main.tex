\documentclass[12pt]{article}

\usepackage{sbc-template}

\usepackage[english]{babel}
\usepackage[utf8]{inputenc}
\usepackage[T1]{fontenc}
\usepackage{amssymb}

\sloppy

\title{Membership in context free languages \\ with the CYK algorithm\footnote{
    As submitted to the INE410113 class (Theory of Computation).}}

\author{Douglas Martins\inst{1}, Emmanuel Podestá Jr.\inst{1}, Gustavo Zambonin\inst{1}}

\address{
  Departamento de Informática e Estatística, Universidade Federal de Santa Catarina \\
  88040-900, Florianópolis, Brazil
  \email{\{marcelino.douglas,emmanuel.podesta,gustavo.zambonin\}@posgrad.ufsc.br}
}

\begin{document} 

\maketitle

\section{Introduction}\label{sec:intro}
Composed by a set of strings of symbols, a formal language can be used as a important instrument to solve problems. We can define a formal language through a grammar, which is a structure able to generate every word in the language. Grammars can be classified as regular, context-free, context-sensitive or even unrestricted, according to the Chomsky hierarchy. The classification of grammars comes from the class of problems that each language generated by each grammar are capable to solve, being regular the most simple class of problems and the unrestricted the most complex class.

One application of formal languages is the description of the programming languages. We can use a context-free grammar to describe their rules, \emph{e.g.} a compiler get as input a source code and uses its grammar to check if the lines of code are syntactically correct, in others words, if the source code is a "member of language". 

The process of analysis a word and check the membership in a language according to the rules of a grammar are called parsing. There exists different types of parsers, according the different types of languages. For the context-free languages, where the grammar are in Chomsky normal formal, we can use the Cocke–Younger–Kasami (CYK) algorithm. 






% Introduce the problem; explain what is the goal of a parsing algorithm; given a context-free grammar $G$ and a word $w$ as input, is it possible to check if $w$ is a member of the language generated by $G$? In our case, consider only grammars in Chomsky normal form.

\section{Cocke–Younger–Kasami algorithm}\label{sec:cyk}

\section{Implementation}\label{sec:imp}


\bibliographystyle{sbc}
\bibliography{sbc-template}

\end{document}
