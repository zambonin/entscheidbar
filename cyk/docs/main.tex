\documentclass[12pt]{article}

\usepackage{sbc-template}

\usepackage[english]{babel}
\usepackage[utf8]{inputenc}
\usepackage[T1]{fontenc}
\usepackage{amssymb}

\sloppy

\title{Membership in context free languages \\ with the CYK algorithm\footnote{
    As submitted to the INE410113 class (Theory of Computation).}}

\author{Douglas Martins\inst{1}, Emmanuel Podestá Jr.\inst{1}, Gustavo Zambonin\inst{1}}

\address{
  Departamento de Informática e Estatística, Universidade Federal de Santa Catarina \\
  88040-900, Florianópolis, Brazil
  \email{\{marcelino.douglas,emmanuel.podesta,gustavo.zambonin\}@posgrad.ufsc.br}
}

\begin{document} 

\maketitle

\section{Introduction}\label{sec:intro}
Composed by a set of strings of symbols with a set of rules, an formal language can be used as a important instrument to solve problems. We can define a formal language through formal grammar, which is a structure able to generate every word of a language. 




Introduce the problem; explain what is the goal of a parsing algorithm; given a context-free grammar $G$ and a word $w$ as input, is it possible to check if $w$ is a member of the language generated by $G$? In our case, consider only grammars in Chomsky normal form.

\section{Cocke–Younger–Kasami algorithm}\label{sec:cyk}

\section{Implementation}\label{sec:imp}


\bibliographystyle{sbc}
\bibliography{sbc-template}

\end{document}
