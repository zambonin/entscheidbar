\documentclass[12pt]{article}

\usepackage{sbc-template}

\usepackage[english]{babel}
\usepackage[utf8]{inputenc}
\usepackage[T1]{fontenc}

\sloppy

\title{Computational Models Equivalent to Turing Machines}

\author{Douglas Martins\inst{1}, Emmanuel Podestá Jr.\inst{1}, Gustavo Zambonin\inst{1}}

\address{
  Departamento de Informática e Estatística, Universidade Federal de Santa Catarina
  \email{gustavo.zambonin@posgrad.ufsc.br}
}

\begin{document} 

\maketitle

\begin{abstract}
Cryptography based on multivariate equations is one of main the approaches for creating algorithms that are quantum-resistant. Nonetheless, digital signature schemes based on these concepts feature impractical key pair sizes, orders of magnitude greater than commonly-used schemes. We identify a special structure on components of the Rainbow signature scheme that allows for the creation of public keys with a partially cyclic construction, reducing storage requirements by up to approximately a factor of three.
\end{abstract}

Written for the INE410111 class (Research Methodology in Computer Science), based on~\cite{Petzoldt:inproc:2010:dec} from~\cite{Petzoldt:phd:2013:jul}. \textbf{This is not a real paper.}

\bibliographystyle{sbc}
\bibliography{sbc-template}

\end{document}
