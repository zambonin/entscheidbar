\documentclass[12pt]{article}

\usepackage{sbc-template}

\usepackage[english]{babel}
\usepackage[utf8]{inputenc}
\usepackage[T1]{fontenc}

\sloppy

\title{Computational Models Equivalent to Turing Machines}

\author{Douglas Martins\inst{1}, Emmanuel Podestá Jr.\inst{1}, Gustavo Zambonin\inst{1}}

\address{
  Departamento de Informática e Estatística, Universidade Federal de Santa Catarina \\
  88040-900, Florianópolis, Brazil
  \email{\{marcelino.douglas,emmanuel.podesta,gustavo.zambonin\}@posgrad.ufsc.br}
}

\begin{document} 

\maketitle

\section{Introduction}\label{sec:intro}

% algorithms are used to solve problems, but not all problems can be solved by algorithms
% to talk about solvable problems is to talk about computable functions
% however, we cannot talk about computable functions without talking about its models of computation, that is, how exactly a set of outputs is computed given a set of inputs, describing memory, units, communications etc
% the main model known is the universal turing machine, and all functions that are computable, are computable with this model
% many others have been created and shown to be equivalent to turing machines
% henceforth we explore curious and/or incommon models of computation that a

\footnote{As submitted to the INE410113 class (Theory of Computation).}

\bibliographystyle{sbc}
\bibliography{sbc-template}

\end{document}
