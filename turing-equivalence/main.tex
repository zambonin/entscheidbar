\documentclass[12pt]{article}

\usepackage{sbc-template}

\usepackage[english]{babel}
\usepackage[utf8]{inputenc}
\usepackage[T1]{fontenc}
\usepackage{amssymb}

\sloppy

\title{Computational Models Equivalent to Turing Machines
\footnote{As submitted to the INE410113 class (Theory of Computation).}}

\author{Douglas Martins\inst{1}, Emmanuel Podestá Jr.\inst{1}, Gustavo Zambonin\inst{1}}

\address{
  Departamento de Informática e Estatística, Universidade Federal de Santa Catarina \\
  88040-900, Florianópolis, Brazil
  \email{\{marcelino.douglas,emmanuel.podesta,gustavo.zambonin\}@posgrad.ufsc.br}
}

\begin{document} 

\maketitle

\section{Introduction}\label{sec:intro}
 
Solving problems is paramount to the advance of society, often being useful to accelerate development in any discipline. By defining a finite number of instructions that may produce outputs fully determined by respective inputs, one creates an algorithm. This historic notion, that which of an ``effective method'' for computations, was subsequently formalised independently by Church~\cite{}, with its $\lambda$-calculus, and Turing, through Turing machines~\cite{}, equivalent models of computation that yield a class of functions known as computable functions. Without loss of generality, these can be thought of as some partial functions on natural numbers, that is, $f : \mathbb{N}^{k} \rightarrow \mathbb{N}$, such that any $k$-tuple will give $f(x)$ through an effective method. 

Note that this computational process may actually never stop, \emph{i.e.} $f(x)$ may be an undefined value. It would be interesting to determine this behaviour for any combination of algorithms --- or ``computer programs'' --- and inputs. Called the halting problem, this was proved to be impossible to solve through the Turing machine model of computation. It is not a computable function, that is, there is no algorithm that accurately determines whether an arbitrary algorithm will halt. The halting problem is frequently featured when models are proven to be equivalent to Turing machines, and will aid to clarify the reasoning behind these.

A Turing machine is defined as a $7$-tuple $M = (Q, \Sigma, \Gamma, \delta, q_{0}, B, F)$, in which $Q$ is a finite set of states, 

% algorithms are used to solve problems, but not all problems can be solved by algorithms
% to talk about solvable problems is to talk about computable functions, general primitive recursive functions and nothing else like I/O or fluff
% however, we cannot talk about computable functions without talking about its models of computation, that is, how exactly a set of outputs is computed given a set of inputs, describing memory, units, communications etc
% the main model known is the universal turing machine, and all functions that are computable, are computable with this model
% the church-turing thesis states that all computing systems capable of general computation are equally powerful
% many others have been created and shown to be equivalent to turing machines
% henceforth we explore curious and/or incommon models of computation that are turing-complete, and consequently, equivalent (talk about difference between equivalence and completeness)
% however note that expressiveness of these models may be limited (how difficult they are to program)
% as such, namely, we will talk about cellular automata, string rewriting systems and Diophantine equations, and how these can be used to construct valid models of computation

% \emph{Organization.} This paper is organized as follows. 

\section{Conclusion}\label{sec:conc}

% in this work we have seen some models of computation that have applications outside theory of computation, like mathematics and biology
% further, these are interchangeable in the sense that one can be represented in the other
% other examples feature lambda calculus, most programming languages (if infinite memory), and even some videogames
% hence, turing completeness is an interesting property of models

\bibliographystyle{sbc}
\bibliography{sbc-template}

\end{document}
